\chapter*{Lista de Abreviações}
\thispagestyle{empty}
%\addcontentsline{toc}{chapter}{Lista de Abreviações}


\begin{acronym}[MOESP-PO] %lembrar que o argumento opcional é o maior acrônimo utilizado %lembrar de manter a lista em ordem alfabética
	\acro{ARX}{modelo autorregressivo com entradas exógenas (\textit{AutoRegressive model with eXogenous inputs})}
	\acro{CVA}{análise de variáveis canônicas (\textit{Canonical Variate Analysis})}
	\acro{EIA-PSO}{otimização por enxame de partículas com estratégia adaptativa, informada e eficaz (\textit{Effective Informed Adaptive} - \textit{Particle Swarm Optimization})}
	\acro{EJL}{estrutura do modelo de Hammerstein proposta por Eskinat, Johnson e Luyben}
	\acro{GPI}{modelo de Prandtl-Ishlinskii generalizado (\textit{Generalized} Prandtl-Ishlinskii)}
	\acro{GRDPI}{modelo de Prandtl-Ishlinskii generalizado e dependente da taxa (\textit{Generalized Rate-Dependent} Prandtl-Ishlinskii)}
	\acro{KU}{estrutura do modelo de Hammerstein proposta por Kortmann e Unbehauen}
	\acro{LIT}{sistema Linear e Invariante no Tempo}
	\acrodefplural{LIT}{sistemaas Lineares e Invariantes do Tempo}
	\acro{LMI}{desigualde matricial linear (\textit{Linear Matrix Inequalitie})}
	\acrodefplural{LMI}{desigualdes matriciais lineares (\textit{Linear Matrix Inequalities})}
	\acro{MATLAB}{\textit{MATrix LABoratory}}
	\acro{MBI}{Modelo de Blocos Interconectados}
	\acrodefplural{MBI}{Modelos de Blocos Interconectados}
	\acro{MIMO}{múltiplas entradas e múltiplas saídas (\textit{Multiple Input Multiple Output})}
	\acro{MISO}{múltiplas entradas e uma saída (\textit{Multiple Input Single Output})}
	\acro{MOESP}{modelo multivariável de erro na saída em espaço de estados (\textit{Multivariable Output-Error State sPace})}
	\acro{MOESP-PO}{modelo multivariável de erro na saída em espaço de estados com entradas passadas atuando como variáveis instrumentais (\textit{Multivariable Output-Error State sPace - Past Output})}
	\acro{MPSO}{otimização por enxame de partículas modificado (\textit{Modified Particle Swarm Optimization})}
	\acro{MQ}{Mínimos Quadrados}
	\acro{N4SID}{Algoritmos numéricos para identificação de sistema em espaço de estados empregando técnicas de subespaço (\textit{Numerical algorithms for Subspace State-Space System IDentification})}
	\acro{NARMAX}{modelo média móvel autoregressivo não linear com entrada exógena (\textit{Nonlinear AutoRegressive Moving Average with eXogenous input})}
	\acro{NARX}{modelo autoregressivo não linear com entrada exógena (\textit{Nonlinear AutoRegressive with eXogenous input})}
	\acro{NF}{\textit{Neuro-Fuzzy}}
	\acro{PBSID}{método de identificação por subespaço baseado em preditor (\textit{Predictor-Based Subspace IDentification method})}
	\acro{PEM}{método de predição de erro (\textit{Prediction Error Method})}
	\acro{P-I}{Prandtl-Ishlinskii}
	\acro{PRBS}{sinal binário pseudoaleatório (\textit{Pseudo Random Binary Signal})}
	\acro{PSO}{otimização por enxame de partículas (\textit{Particle Swarm Optimization})}
	\acro{RDPI}{modelo de Prandtl-Ishlinskii dependente da taxa (\textit{Rate-Dependent} Prandtl-Ishlinskii)}
	\acro{RMSE}{raiz quadrada do erro médio quadrático (\textit{Root Mean Square Error})}
	\acro{SBAI}{Simpósio Brasileiro de Automação Inteligente}
	\acro{SIMO}{uma entrada e múltiplas saídas (\textit{Single Input Multiple Output})}
	\acro{SISO}{uma entrada e uma saída (\textit{Single Input Single Output})}
	\acro{SNR}{relação sinal ruído (\textit{Signal Noise Ratio})}
	\acro{SVD}{decomposição em valores singulares (\textit{Singular Value Decomposition})}
\end{acronym}
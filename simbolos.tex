\chapter*{Lista de Símbolos}
\thispagestyle{empty}
%\addcontentsline{toc}{chapter}{Lista de Símbolos}


\begin{itemize}[leftmargin=30pt,labelsep=1em,align=left]
	\setlength\itemsep{0em}
	\item[$\in$] Pertence;
	\item[$\mathbb{R}$] Conjunto dos número reais;
	\item[$p$] Quantidade de entradas de processos ou modelos de blocos interconectados;
	\item[$k$] Variável de tempo discreto;
	\item[$u(k)$] Entrada de sistemas não autônomos no instante $k$, $u(k) \in \mathds{R}^{p}$;
	\item[$\varrho$] Quantidade de sinais intermediários de processos ou modelos de blocos interconectados;
	\item[${v}(k)$] Sinal intermediário dos modelos de blocos interconectados no instante $k$, $v(k)~\in~\mathds{R}^{\varrho}$;
	\item[$m$] Quantidade de saídas de processos ou modelos de blocos interconectados;
	\item[$y(k)$] Sinal de saída dos modelos no instante $k$, $y(k)  \in \mathds{R}^{m}$;
	\item[$\mathcal{N}$] Não linearidades do processo ou do modelo de blocos interconectados;
	\item[$\mathcal{L}$] Parcela linear do processo ou do modelo de blocos interconectados;
	\item[$x(k)$] Vetor de estados no instante $k$, $x(k)  \in \mathds{R}^{n}$;
	\item[$A$] Matriz da dinâmica do sistema $ \in \mathds{R}^{n \times n}$;
	\item[$B$] Matriz de entradas $ \in \mathds{R}^{n \times \varrho}$;
	\item[$C$] Matriz de saída $ \in \mathds{R}^{m \times n}$;
	\item[$D$] Matriz de transmissão direta $ \in \mathds{R}^{m \times \varrho}$;
	\item[$w(k)$] Vetor de ruído de medição $ \in \mathds{R}^{n}$;
	\item[$\mathcal{T}$] Transformação de similaridade;
	\item[$\hat{(\bullet)}$] Estimativa de $(\bullet)$;
	\item[$i$] Índice que assume valores de 1 a $p$;
	\item[$\textrm{\footnotesize \cursive \textit{f}}\,(\cdot)$] Função crisp;
	\item[$\vartheta$] Parâmetro consequente do modelo Takagi-Sugeno de ordem zero;
	\item[$c$] Centro de uma função de pertinência gaussiana;
	\item[$\sigma$] Dispersão de uma função de pertinência gaussiana;
	\item[$t$] Variável de tempo contínuo;
	\item[$b_\bullet$] Parâmetros do modelo de Bouc-Wen;
	\item[$\dot{\bullet}$] Derivada de $\bullet$ no tempo contínuo;
	\item[$d_\bullet$] Parâmetros do modelo de Duhem;
	\item[$p_\bullet$] Disparo do operador elementar de Preisach ;
	\item[$N_P$] Quantidade de operadores no modelo de Prandtl-Ishlinskii;
	\item[$\mathcal{S}$] Operador elementar \textit{stop};
	\item[$r$] Disparo de um operador elementar de Prandtl-Ishlinskii;
	\item[$j$] Índice que assume valores de 1 a $N_P$;
	\item[$\theta$] Parâmetro dos operadores do modelo de Prandtl-Ishlinskii;
	\item[$\mathcal{P}$] Operador elementar \textit{play};
	\item[$\mathcal{B}$] Função de borda do operador \textit{play};
	\item[$\mathcal{G}$] Operador \textit{play} generalizado;
	\item[$\mathfrak{b}$] Parâmetros da função de borda;
	\item[$\breve{\theta}$] Parâmetro dos disparos dinâmicos do modelo de Prandtl-Ishlinskii;
	\item[$T_s$] Período de amostragem;
	\item[$p_n$] Ordem de persistência de excitação;
	\item[$P_N$] Número de colunas da matriz $\mathcal{V}$;
	\item[$h$] Quantidade de blocos linha de uma matriz em blocos de Hankel;
	\item[$\Gamma$] Matriz de observabilidade estendida;
	\item[$T$] Transposto;
	\item[$N_L$] Quantidade de dados para identificar dinâmica linear;
	\item[$\mathscr{V}$] Matriz em bloco de Hankel equacionada com $v(k)$;
	\item[$\Upsilon$] Matriz em bloco de Hankel equacionada com $y(k)$;
	\item[$R$] Matriz à esquerda da fatoração RQ;
	\item[$Q$] Matriz à direita da fatoração RQ;
	\item[$U_o$] Matriz à esquerda da decomposição em valores singulares;
	\item[$S_o$] Matriz com valores singulares da decomposição em valores singulares;
	\item[$V_o$] Matriz à direita da decomposição em valores singulares;
	\item[$l$] Índice que assume valores de 1 a $m$;
	\item[$N_N$] Quantidade de dados para identificar a não linearidade;
	\item[$G_\mathcal{N}$] Ganho em regime de uma parcela estática não linear do modelo de Hammerstein;
	\item[$G_\mathcal{L}$] Ganho em regime de uma parcela dinâmica linear do modelo de Hammerstein;
	\item[$G_\mathcal{NL}$] Ganho em regime de um modelo de Hammerstein;
	\item[$\beta$] Ganho de uma função estática não linear no ponto $u_\beta$;
	\item[$I_\bullet$] Matriz identidade de ordem $\bullet$; 
	\item[$N_N^{(i)}$] Quantidade de dados do ensaio estático realizado na entrada $i$;
	\item[$\bar{N}_{i,j}$] Quantidade de dados formando o $j$-ésimo \textit{cluster} da entrada $i$;
	\item[$N_i$] Quantidade total de \textit{clusters} para a entrada $i$;
	\item[$\mathscr{J}$] Indica a posição do \textit{cluster} mais próximo do dado $u_i(k)$;
	\item[$S_{i,0}$] Abrangência do \textit{cluster} $i$;
	\item[$\lambda$] Fator de deslocamento do centro do \textit{cluster};
	\item[$\rho$] Controle da dispersão da gaussiana;
	\item[$E$] Função custo da metodologia \textit{neuro-fuzzy};
	\item[$\Phi$] Ponderação utilizada na defuzificação;
	\item[$\mu$] Função gaussiana;
	\item[$n_d$] Ordem a partir da qual a derivada parcial de $E$ é nula;
	\item[$n_p$] Número de estágios do registrador para gerar um sinal binário pseudoaleatório;
	\item[$\mathbb{N}$] Conjunto dos números naturais;
	\item[$\mathscr{U}$] Distribuição uniforme;
	\item[$\mu_u$] Média da distribuição uniforme;
	\item[$\sigma_u$] Desvio padrão da distribuição uniforme;
	\item[$\mathscr{N}$] Distribuição normal;
	\item[$\mu_n$] Média da distribuição normal;
	\item[$\sigma_n$] Desvio padrão da distribuição normal;
	\item[$\iota$] Índice que assume valores de 1 a $\varrho$;
	\item[$\mathcal{GR}$] Operador \textit{play} generalizad dependente da taxa;
	\item[$\Theta$] Parâmetros do operador $\mathcal{GR}$;
	\item[$N_V$] Quantidade total de parâmetros de $\Theta$;
	\item[$F$] Função de aptidão das partículas de uma nuvem;
	\item[$V$] Matriz com a velocidade das partículas de uma nuvem;
	\item[$\tilde{\theta}$] Parâmetros que controlam a função de borda do operador generalizado dependente da taxa de Prandtl-Ishlinkii;
	\item[$\epsilon$] Parâmetro interno do modelo GRDPI;
	\item[$\Delta u$] Variação entre amostras do sinal de entrada;
	\item[$N_G$] Quantidade de dados usados para ajustar o ganho total do modelo de Hammerstein MIMO GRDPI;
	\item[$\xi$] Quantidade de partículas selecionadas para compor o grupo de melhores partículas no PSO;
	\item[$K_{\mu}$] Quantidade de iteração que iterações consecutivas sem alteração da função custo $F(\Theta)$ a partir da qual podem ocorrer mutações;
	\item[$P$] Quantidade de partículas do PSO;
	\item[$\ell$] Índice que assume valores de 1 a $P$;
	\item[$\kappa$] Índice da atual iteração do PSO;
	\item[$K_{\max}$] Quantidade de iterações que o PSO é executado;
	\item[$\alpha$] Probabilidade de ocorrer mutação no PSO;
	\item[$\varpi$] Taxa de aprendizado cognitivo e social do PSO;
	\item[$U$] Matriz com amostras da distribuição uniforme;
	\item[$\odot$] Multiplicação elemento a elemento;
	\item[$M^{(p)}$] Média ponderada das melhores posições pessoais;
	\item[$M^{(g)}$] Média ponderada das melhores posições globais;
	\item[$\varsigma$] Média ponderada das melhores funções de aptidão;
	\item[$q$] Índice que assume valores de 1 a $\xi$ independente de $g$;
	\item[$g$] Índice que assume valores de 1 a $\xi$ independente de $q$;
	\item[$s$] Índice que assume valores de 2 a $P$;
	\item[$\chi$] Fator de constrição do PSO;
	\item[$\nu$] Amostra de uma distribuição uniforme;
	\item[$\gamma$] Uma dimensão aleatória de velocidade $V$;
	\item[$\eta$] Índice aleatório que indica uma das $P$ partículas da nuvem;
	\item[$\zeta$] Fator aleatório que altera a velocidade de uma partícula;
	\item[$\overline{b}$] Nível superior de um sinal PRBS;
	\item[$\underline{b}$] Nível inferior de um sinal PRBS;
	\item[$\overline{\theta}$] Ganhos da parte superior do sinal intermediário;
	\item[$\underline{\theta}$] Ganhos da parte inferior do sinal intermediário;
	\item[$\underline{\Theta}$] Matriz de parâmetros com ganhos do sinal intermediário a serem otimizados;
	\item[$\underline{v}$] sinal intermediário compensado com os ganhos $\underline{\Theta}$;
	\item[$\underline{y}$] sinal intermediário compensado com os ganhos $\underline{\Theta}$;
	\item[$\underline{\mathcal{L}}$] Estimativa inicial do bloco dinâmico linear.
\end{itemize}

